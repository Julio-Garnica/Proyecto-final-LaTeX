\documentclass[a4paper,10pt]{article}  
\usepackage[T1]{fontenc}              
\usepackage{amsmath,amssymb,amsthm}   
\usepackage{lmodern,microtype,bm}    
\usepackage{graphicx}
\usepackage[utf8]{inputenc}           
\usepackage[english,spanish]{babel}   
\usepackage[style=mexican]{csquotes}  
\usepackage[a4paper, total={6in, 8in}]{geometry} 
\topmargin = -50pt                    
\textwidth = 450pt                   
\textheight = 675pt                   
\hoffset = -10pt                     
\setlength{\parindent}{2em}           
\setlength{\parskip}{0.5em}           


\title{\huge{Épsilons, deltas y la invención de los números reales.}}
\author{Julio César Lozano Garnica}
\date{04 de junio del 2025}

\begin {document}

\maketitle

Este es un ensayo sobre la sección titulada \textit{Épsilons, deltas y la invención de los números reales} que forma parte del capítulo \textit{Cálculo / Análisis}, del libro \enquote{El lenguaje de las matemáticas} escrito por Raúl Rojas Gonzaléz. 

\section*{Épsilons y deltas.}

Me parece muy curioso como es que surge la necesidad de ponerle nombre a un concepto tan abierto y estricto a la vez como es el caso de las épsilons $\varepsilon$ y las deltas $\delta$. Como menciona el autor, en ambos casos se trata de definir un intervalo, del cual una sucesión convergente, un límite o la continuidad de una función no tienen la posibilidad de escapar si es que tienen la intención de existir. Que nos refiramos a estas ideas con una sola letra y además sean letras griegas que a mi parecer poseen de una gran estética, me parece elegante y muy concreto, pues al trabajar con ellas es muy práctico referirnos únicamente a una épsilon, en vez de enunciar a un concepto largo de díficil digestión mental. En este sentido, creo que tanto el concepto mismo, como la idea de  referirnos a él, demuestran la gran genialidad de las personas que los inventaron.

En cuanto a como es que el autor nos presenta la información sobre este tema, me parece adecuado; pues a pesar de exponer únicamente los conceptos de manera intuitiva y resumida, considero que lográ pintar en nuestras mentes una primera idea muy sólida de lo que es tratar con estos \enquote{intervalos} y de lo que se busca llegar con ellos. Hablando desde mi experiencia personal, haber obtenido un gráfico como lo es la \textit{ Figura 1: Una sucesión convergente}, en mi primer intento por cursar Cálculo I, hubiera sido sin duda, una experiencia mucho mejor a lo que fue tratar de abstraer sin éxito un concepto al que nunca había estado expuesto y que no genero en mi otra cosa que no fuera frustración y decepción. Sentimiento que al parecer es muy común, pues el mismo autor refiere a este tema como \enquote{aterrorizante} para todos aquellos que nos aventuramos a cursar los primeros semestres de la carrera de matemáticas.

\begin{figure}[h]
    \centering
    \includegraphics[width=0.5\linewidth]{FIGURA V.14. Una sucesión convergente}
    \caption{Una sucesión convergente}
    \label{fig:sucesion_convergente}
\end{figure}

\section*{Invención de los números reales.}

Más adelante en el texto, el autor da una pequeña introducción a la historia de la invención de los números reales. Por decir pequeña, tal vez me este quedando muy corto, pues comparada con el libro \enquote{Un acercamiento a los fundamentos del Cálculo} del profesor Fernández, no es nada, sin embargo; rescata aspectos muy importantes como lo son los números irracionales y me gusta mucho como es que menciona estos huecos que existirían en la recta de los números reales si es que estos no existieran y la razón de mi agrado, es que teniendo esto en mente, uno como estudiante puede empatizar mucho más con la urgencia de llenar esos huecos y puede celebrar con gusto los éxitos de las aportaciones de Cauchy, Dedekind y Cantor en sus creaciones de modelos para describir a los números reales de una manera formal y \textit{completa}.

\begin{figure}[h]
    \centering
    \includegraphics[width=0.5\linewidth]{Recta de los números reales}
    \caption{Recta de los números reales}
    \label{fig:recta_reales}
\end{figure}

Sin embargo, considero que el autor comete un ligero error al utilizar su ejemplo respecto a la definición de los números reales, haciendo analogía con las carreteras que desembocan en el mismo lugar y mencionando que en el plano euclidiano no se quiere que hayan agujeros. Me parece que se enreda en su idea y no resulta tan explicativo como se propone, pues el concepto de plano Euclidiano no es tan popular a comparación de si solo se intenta imaginar una línea recta punteada. 

En general, me parece muy adecuado la manera en la que el autor aborda los temas referentes a esta sección. Pues si bien recupera los aspectos históricos referentes a las épsilons y las deltas, también se permite presentar en la mente del lector las ideas intuitivas que motivaron la existencia de estos conceptos y eso me parece indispensable que este presente en cualquier texto divulgativo de matemáticas. 

\end {document}