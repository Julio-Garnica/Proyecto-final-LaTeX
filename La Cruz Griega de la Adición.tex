\documentclass[11pt, a4paper]{article} % Tamaño de fuente base y tipo de papel
\usepackage[spanish]{babel} % Soporte para idioma español
\usepackage[utf8]{inputenc} % Codificación de entrada (para pdflatex)
\usepackage{lmodern} % Usa la versión Latin Modern de la fuente Computer Modern, que es más moderna y nítida.
% Opcional, para usar fuentes del sistema (requiere XeLaTeX o LuaLaTeX):
% \usepackage{fontspec}
% \setmainfont{Palatino Linotype} % Intenta con Palatino Linotype si la tienes
% \setsansfont{Calibri} % Para sans-serif
% \usepackage{microtype} % Para mejor justificación y espaciado (microtipografía)

\usepackage{geometry} % Para controlar los márgenes
\geometry{a4paper,
  left=1.2in,
  right=1.2in,
  top=1.2in,
  bottom=1.2in
}

\usepackage{titlesec} % Para el formato de títulos
\usepackage{fancyhdr} % Para cabeceras y pies de página
\usepackage{setspace} % Para el interlineado

% Configuración de la paginación para el pie de página
\pagestyle{fancy}
\fancyhf{} % Limpia la cabecera y el pie de página
\renewcommand{\headrulewidth}{0pt} % Sin línea en la cabecera
\fancyfoot[C]{\thepage} % Número de página centrado en el pie

% Ajuste de interlineado para el cuerpo del texto
\onehalfspacing % Equivalente a 1.5 líneas, proporciona más "aire". Puedes usar \linespread{1.15} si lo prefieres más denso.

% --- Estilo del Título del Ensayo ---
% Un título centralizado, con tamaño y formato específicos
\newcommand{\ensayoTitulo}[2]{%
  \begin{center}
    \vspace*{1cm} % Espacio adicional antes del título
    {\fontsize{24}{26}\selectfont\textbf{#1}} \\ % Título principal
    \vspace{0.3cm} % Espacio entre título principal y subtítulo
    {\fontsize{16}{18}\selectfont\textit{#2}} % Subtítulo
    \vspace*{1.5cm} % Espacio después del subtítulo
  \end{center}
}

\begin{document}

% Llamada al comando del título del ensayo
\ensayoTitulo{La Cruz Griega de la Adición:}{Un Símbolo con Historia y Significado}

En el vasto universo de las matemáticas, pocos símbolos son tan omnipresentes y fundamentales como la simple cruz que denota la adición. Más allá de su aparente obviedad en nuestro uso cotidiano, la ``cruz griega'' de la adición, o el signo de más (+), encierra una fascinante historia que se entrelaza con el desarrollo del pensamiento matemático y la necesidad humana de abstraer y simplificar el mundo. Este humilde grafismo, que hoy damos por sentado, es el resultado de una evolución histórica que transformó una abreviatura en un universal icónico.

Antes de la estandarización del signo de más, la adición se expresaba de maneras diversas y a menudo engorrosas. En la antigüedad, las sumas se indicaban con palabras completas, como ``y'' o ``plus'', o mediante abreviaturas manuscritas que variaban según el escriba y la región. El desafío residía en encontrar una notación concisa y unívoca que trascendiera las barreras lingüísticas y facilitara los cálculos, liberando la mente de la carga de la transcripción literal.

La historia más aceptada del origen del signo de más nos remonta al siglo XV, a los trabajos de matemáticos europeos, especialmente alemanes. Se cree que el símbolo ``+'' surgió de la abreviatura de la palabra latina ``et'' (que significa ``y''), la cual era comúnmente utilizada para indicar una suma. Los escribas, en su afán por la eficiencia, habrían simplificado gradualmente la ``t'' estilizada hasta transformarla en la cruz que hoy conocemos. Esta ``t'' abreviada, a menudo con una barra horizontal encima, se convirtió en una cruz, marcando un hito en la notación matemática. Johannes Widmann, en su libro de aritmética comercial \textit{Mercantile Arithmetic} de 1489, es a menudo acreditado como el primero en usar los signos ``+'' y ``-'' en un contexto impreso, aunque su uso aún no estaba completamente generalizado ni universalmente aceptado.

La adopción de este símbolo no fue inmediata ni uniforme. Durante un tiempo, coexistieron diversas notaciones en diferentes regiones y entre distintos matemáticos. Sin embargo, la sencillez visual y la economía de espacio de la cruz la hicieron irresistible. Su claridad y la facilidad con la que podía ser reproducida, tanto en manuscritos como en las imprentas emergentes, contribuyeron a su rápida difusión y eventual dominio. Al mismo tiempo, el signo de menos (-) también se estaba estandarizando, a menudo derivado de la barra horizontal que denotaba la sustracción o la ausencia de un elemento.

La cruz de la adición es más que un mero atajo; es un testimonio de la progresiva abstracción en las matemáticas. Al encapsular la acción de ``agregar'' o ``combinar'' en un único y simple símbolo, se liberó la cognición para enfocarse en la estructura de los problemas y en la relación entre las cantidades, en lugar de en la operación misma. Es un ejemplo perfecto de cómo el lenguaje de las matemáticas se vuelve más eficiente y universal a través de la notación.

En conclusión, la ``cruz griega'' de la adición, lejos de ser un invento trivial, es un monumento a la evolución del pensamiento humano. Su historia nos recuerda que incluso los elementos más básicos del lenguaje matemático tienen profundas raíces culturales e históricas, y que la búsqueda de la claridad y la concisión ha sido una constante impulsora de la innovación en este fascinante campo. Es un símbolo que, con cada uso, nos conecta con siglos de ingenio y simplificación.

\end{document}