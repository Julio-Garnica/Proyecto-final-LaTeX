\documentclass[12pt]{article}
\usepackage[spanish]{babel}
\usepackage{lmodern}
\renewcommand{\familydefault}{sfdefault}
\usepackage{geometry}
\usepackage{setspace}
\usepackage{graphicx}
\usepackage{xcolor}
\definecolor{dorado}{RGB}{255,199,51}
\definecolor{azul2}{RGB}{15,29,145}
\usepackage{tikz}

\begin{document}

	\begin{titlepage}
    
		\thispagestyle{empty}
        
		\newgeometry{left=2cm, right=1cm, top=2cm, bottom=2cm}
        
		\begin{tikzpicture}[overlay, fill opacity=0.8]
			\rotatebox{-45}{\fill [azul2] (11.7,-23) rectangle (21,20);}
			\fill [dorado] (15,-27) rectangle (16.5,4);
		\end{tikzpicture}
        
		\begin{spacing}{1.5}
			{\huge \bfseries \noindent Universidad Nacional Autónoma de México}\\ [5pt]
			{\Large Cálculo Diferencial e Integral II}\\ [5pt]
			{\Large Proyecto final de \LaTeX}
		\end{spacing}
		
		\begin{minipage}{2cm}
			\hspace{0.5cm}\rotatebox{90}{{\Large}}
		\end{minipage} \hfill
		\begin{minipage}{2.4cm}
			\begin{spacing}{5}
				{\color{black}\fontsize{60}{70} \selectfont
                \bfseries 2\\0\\2\\5}
			\end{spacing}
		\end{minipage}
        
		\vfill
        
		\begin{flushleft}
			{\Large \color{white}
			\textbf{Armenta Cortes Pamela} \\ [5pt]
			\textbf{Lozano Garnica Julio César} \\ [5pt]
			\textbf{Perez Almazán Marcos} \\ [5pt]}
		\end{flushleft}
        
		\vspace{1cm}
            
	\end{titlepage}
		
\end{document}  